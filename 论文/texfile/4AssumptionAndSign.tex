\newpage%\新一页
\section{模型的假设}
为了构建更为精确的数学模型,本文根据实际情况做出以下合理的假设或条件约束:
%1.	假设题目所给的数据真实可靠;
%注意:假设对整篇文章具有指导性,有时决定问题的难易。一定要注意假设的某种角度上的合理性,不能乱编,完全偏离事实或与题目要求相抵触。注意罗列要工整。
\begin{enumerate}
	\item 乙醇催化偶合制备 \( \text{C}_4 \) 烯烃的过程中,除给定催化剂组合以及温度和反应时间外,其它条件对实验结果的影响可忽略不计。
	\item 在反应进行时,实验室内环境温度和气压不会对反应造成影响。
	\item 实验设备完好且在实验过程中不会出现漏气、破损等情况导致实验失败或数据出现差错。
	\item 在实验条件下,逆反应速率极慢,且 \( \text{C}_4 \) 烯烃生成后迅速脱离反应体系,从而逆反应影响可以忽略。
\end{enumerate}

\section{符号说明}
\begin{table}[H] %[h]表示在此处添加浮动体,默认为tbf,即页面顶部、底部和空白处添加
	\captionsetup{skip=4pt} % 设置标题与表格的间距为4pt
	\centering
	\setlength{\arrayrulewidth}{1pt} % 设置表格线条宽度为1pt
	\begin{tabular}{cc} %c表示居中,l表示左对齐,r表示右对齐,中间添加“|”表示竖线
		\hline
		\makebox[0.15\textwidth][c]{符号} & \makebox[0.6\textwidth][c]{说明}  \\ 
		\hline
		$X$ & 乙醇转化率 \\
		$S$ & C4烯烃选择性 \\
		$Y$ & C4烯烃收率($Y = X \times S$) \\
		$r$ & 皮尔逊相关系数 \\
		$\lambda$ & 岭回归正则化参数(惩罚系数) \\
		$\alpha$ & 岭回归模型通过交叉验证得到的最优正则化参数 \\
		$w$ & 岭回归模型的回归系数向量 \\
		$R^2$ & 决定系数,用于评估模型拟合优度 \\
		$n$ & 样本数量 \\
		$i, j$ & 求和或循环中的索引变量 \\
		$\bar{x}, \bar{y}$ & 变量 $x$ 和 $y$ 的均值 \\
		\hline
	\end{tabular}
	% \hline是横线,采用\makebox设置列宽
\end{table}


