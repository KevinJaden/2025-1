\maketitle%与上面的\title对应
\begin{abstract}%摘要+摘要+摘要+摘要+摘要

	
	 \( \text{C}_4 \) 烯烃是化学工业生产与医药制造的重要基础化工原料,也是石油化工产业的基础。近年来,随着国内化工产业的不断进步, \( \text{C}_4 \) 烯烃的综合利用越来越受到重视,乙醇催化偶合制备 \( \text{C}_4 \) 烯烃逐渐进入人们的视野。因此,选择合适的催化生产工艺,实现稳定、高效生产的目标,不仅能为企业带来明显的经济效益,同时也能促进我国相关化工企业的发展。
	
	%第一段——问题重述+简要思想:首先简要叙述所给问题的背景和动机,并分别分析每个小问题的特点(以下以三个问题为例)。根据这些特点说出自己的思想:针对于问题1,采用。。。。。。。。的方法解决;针对问题2用。。。。。。。。的方法解决;针对问题3用。。。。。。。。的方法解决。
	
	
	
	%\textbf{}:加粗	
	\textbf{针对问题一},	本文首先完成全局数据清洗预处理,通过\textbf{三次样条插值}完成缺失值填补,接着对每一个催化剂组合和温度绘制折线图对比乙醇转化率和 \( \text{C}_4 \) 烯烃选择性的趋势图分析变量与温度间的关系为强正相关,又通过皮尔逊相关系数计算,可筛选出相关性最强的组合作为工艺优化对象。又对350度时给定的催化剂组合所得实验数据进行数据可视化绘制折线图,观察可得,在一定温度和相同催化剂下,随着时间的增加,反应物乙醇转化率持续下降,逐渐趋于稳定;\( \text{C}_4 \) 烯烃选择性增加,并趋于稳定,与时间呈现出弱负相关性。
	%第二段——模型建立及求解结果:介绍思想和模型: 对于问题1我们首先建立了。。。。。。。。模型I。首先利用。。。。。。,其次计算了。。。。。。,并借助。。。。。。数学算法和。。。。。。软件得出了。。。。。。结论。
	
	
	
	
	\textbf{针对问题二},依据问题一所得信息,使用\textbf{正则表达式}将催化剂组合和温度信息分离构建为独立变量,通过控制变量法分别控制装料方式不同、乙醇浓度不同、Co负载量不同、有无HAP、总质量不变,Co与HAP质量比不同以及Co与HAP质量比1:1,总质量不同进行实验对比可得,温度对两因变量影响最强且不易受干扰,其他因素都对因变量影响较弱但不可忽略。
	%(第3段)	对于问题2我们用。。。。。。。。
	
	
	
	\textbf{针对问题三},探究多个自变量与在相同的实验条件下,如何组合从而使得 \( \text{C}_4 \) 烯烃的收率最大。通过题中所给的 \( \text{C}_4 \) 烯烃收率计算式,写出全数据收率结果,在此基础上构建\textbf{多项式回归},训练成功后使用\textbf{网格搜索}得到温度限制在450°C时, \( \text{C}_4 \) 烯烃最大收率为52.03\%;当温度限制在350°C时, \( \text{C}_4 \) 烯烃最大收率为21.70\%。
	%(第4段)	对于问题3我们用。。。。。。。。(模型的建立与求解结果的陈述中,思想、模型、软件和结果必须描述清晰,亮点详细说明需突出。针对不同问题可独立成段也可采用一段式仅用分号“;”分割,摘要只接受文字描述形式,不接受图表等其他方式)
	
	\textbf{针对问题四},基于前三问模型结论,采用模型驱动的序贯实验设计策略精准规划五次实验:首先在450℃(全局最优预测点)和350℃(低温最优点)执行确认实验,验证模型外推可靠性;其次依据岭回归分析(Co负载量与选择性负相关),在A3基础上将Co负载量提升至8wt\%,实施极值寻优以探测参数拐点;针对催化剂失活问题,第四组延长反应时间至400分钟,构建加速寿命测试量化失活动力学;第五组在B装料方式下复现高温条件,通过对照实验检验工艺泛化能力。设计遵循"模型预测→靶向实验→反馈修正"动态闭环,五次实验分别覆盖参数验证、边界探索、稳定性评估及系统适应性检验,以单次实验信息增量最大化为目标,在有限次数内高效逼近最优工艺,显著提升研究效率与科学性。
	
	
	
	%(第5段)	优化结果及总结:在。。。。。。条件下,针对。。。。。。模型进行适当修改与优化,这种条件的改变可能来自你的一种猜想或建议。要注意合理性。此推广模型可以不深入研究,也可以没有具体结果。
	
	%注:字数300~600之间,需控制在一页;摘要中必须将具体方法、模型和所得结果写出来;摘要要求“总分总”,段开头可用“针对问题1,针对问题2,针对问题3..”或者“首先,然后,其次,最后”等词语进行有逻辑的论述。摘要是重中之重,必须严格执行!
	
	
	
	\keywords{\textbf{三次样条插值}\quad  \textbf{正则表达式}\quad   \textbf{岭回归}\quad   \textbf{多项式回归}\quad \textbf{网格搜索}}
	%\keywords:关键词;\quad:空格
	%使用到的模型名称、方法名称、特别是亮点一定要在关键字里出现,3~5个较合适,用分号隔开
\end{abstract}
