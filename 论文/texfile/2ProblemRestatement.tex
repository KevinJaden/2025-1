%\section{}-subsection{}-subsubsection{}:标题1-3级
\section{问题背景与重述}
\subsection{问题背景}
%在保持原题主体思想不变下,可以自己组织词句对问题进行描述,主要数据可以直接复制,对所提出的问题部分基本原样复制。篇幅建议不要超过一页。大部分文字提炼自原题。)
 \( \text{C}_4 \) 烯烃可用于生产聚丙烯等多项化工产品,是重要的基础化工原料,是优质的汽油调和组分,是石油化工产业的基础。近些年,随着国内煤化工业的不断发展, \( \text{C}_4 \) 烯烃的综合利用越来越受到重视,乙醇催化偶合制备 \( \text{C}_4 \) 烯烃逐渐进入人们的视野。

因此,选择合适的催化生产工艺,达到稳定、高效生产的目的,提高 \( \text{C}_4 \) 烯烃的选择性和收率,不仅能为企业带来明显的经济效益,同时也能促进我国相关化工企业的发展。







\subsection{问题重述}
%结合以上情况,建立数学模型解决以下问题:
某化工实验室针对不同催化剂在不同温度下做了一组实验,结果如附件 1 和附件 2 所示。请通过数学建模完成下列问题:

\begin{enumerate}
	\item 问题一:针对附件 1 所给的催化剂组合,研究其乙醇转化率、 \( \text{C}_4 \) 烯烃的选择性与温度的关系;并分析附件 2 所给定的催化剂组合在一次实验中 350℃ 下不同时间的实验结果进行分析。
	
	\item 问题二:研究不同催化剂组合和温度对乙醇转化率和 \( \text{C}_4 \) 烯烃选择性的影响。
	
	\item 问题三:如何选择催化剂组合与温度,使得在相同实验条件下 \( \text{C}_4 \) 烯烃收率尽可能高?若使温度低于 350 度,又如何选择催化剂组合与温度,使得 \( \text{C}_4 \) 烯烃收率尽可能高?
	
	\item 问题四:如果允许再增加五次实验,应如何设计,并给出详细理由。
\end{enumerate}