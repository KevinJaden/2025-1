\newpage
\section{问题分析}
%TOPS:这部分写差不多一页(删掉此句话)
\subsection{问题一分析}
对于问题一,第一部分的问题,可以分别针对两种装料方式的十九中不同催化剂组合所得实验结果绘制折线图,根据折线图及分析可得各催化剂组合温度和 \( \text{C}_4 \) 烯烃选择性以及乙醇转化率的关系,同时可计算变量之间的相关性,绘制分布气泡图。第二部分的问题根据附件二所给额外实验数据做分析,绘制折线图,由于其他变量都是由于催化剂组合不同和温度不同造成的结果,不需要做主成分分析进行降维,直接分析即可。



\subsection{问题二分析} 
对于问题二,要求探讨不同催化剂组合及温度对乙醇转化率以及 \( \text{C}_4 \) 烯烃选择性大小的影响,可以通过控制变量法分组进行比较,如装料方式不同、乙醇浓度不同、Co负载量不同、有无HAP、Co与HAP质量比不同以及总质量不同等可控变量,进行分组后分别比较。为了弄清催化剂类型对他们的影响,可以对催化剂进行文本数据提取,并将 \( \text{C}_4 \) 烯烃选择性和乙醇转化率分别作为因变量,以催化剂数据作为自变量构建岭回归得到相关关系。



\subsection{问题三分析}
对于问题三,要选择使得 \( \text{C}_4 \) 烯烃收率达到最高的催化剂组合和温度,我们可以在问题二拆解催化剂数据的基础上首先全局选取收率最高的催化剂组合再控制温度在350度以下选取局部收率最高的催化剂组合。由于这样只是选出了在数据面上的优选项,我们可以通过建立多项式回归模型,在连续的参数空间内进行网格搜索,以预测理论上可达到的最高C4烯烃收率及其对应的催化剂组合和温度。




\subsection{问题四分析}
对于问题四,综合前三问所的结果,考虑在催化剂组合和温度的各个参数变化方向上进行调整,选出最合适的五个实验环境,最大化信息获取 ,从而更有效地逼近最优的工艺条件。


